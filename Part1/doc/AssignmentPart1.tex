\documentclass[11pt]{article}
\usepackage{a4wide}
\usepackage{latexsym}
\usepackage{amssymb}
\usepackage{epic}
\usepackage{graphicx}
%\pagestyle{empty}
\newcommand{\tr}{\mbox{\sf true}}
\newcommand{\fa}{\mbox{\sf false}}
\newcommand{\bimp}{\leftrightarrow}


\begin{document}

\section*{
\begin{center}
Practical Assignment Part 1\\
Automated Reasoning 2IMF25\\
\end{center}
}

\begin{center}
Technische Universiteit Eindhoven\\
Jiahuan Zhang (j.4.zhang@student.tue.nl)\\
Hector Joao Rivera Verduzco (h.j.rivera.verduzco@student.tue.nl)\\

\end{center}
\date{\today}
\vspace{5mm}

\section*{Problem 1}

Six trucks have to deliver pallets of obscure building blocks to a magic factory. Every truck has a capacity of 7800 kg and can carry at most eight pallets. In total, the following has to be delivered:
\begin{itemize}
  \item Four pallets of nuzzles, each of weight 700 kg.
  \item A number of pallets of prittles, each of weight 800 kg.
  \item Eight pallets of skipples, each of weight 1000 kg.
  \item Ten pallets of crottles, each of weight 1500 kg.
  \item Five pallets of dupples, each of weight 100 kg.
\end{itemize}
Prittles and crottles are an explosive combination: they are not allowed to be put in the same truck.\\
Skipples need to be cooled; only two of the six trucks have facility for cooling skipples.\\
Dupples are very valuable; to distribute the risk of loss no two pallets of dupples may be in the same truck.\\
Investigate what is the maximum number of pallets of prittles that can be delivered, and show how for that number all pallets may be divided over the six trucks.

\vspace{4mm}

\subsection*{Solution:}

We generalize the problem to offer $m$ trucks to deliver the pallets. $m$ is a positive integer. Then we introduce some variables $t_{ij}$ for $i = 1, ..., m$ and $j = 1,...,5$, which represents the number of pallets of obscure building blocks $j$ on the $i$-th truck. $t_{ij}$ is a nature number. Every truck has a capacity, which is denoted as $C$, and the maximum number of pallets each truck can carry is $M$ kg.

The pallets of the same building blocks have the same weight, $weight(j)$. Each kind of building blocks has a predefined number, $\sharp pallets(j)$.

Now we consider the conditions for the delivery.

\begin{description}
  \item[Condition 1:] $t_{ij}$ should be no less than $0$.

  This is expressed by the formula
  \[ \bigwedge_{i,j:1 \leq i \leq m \wedge 1 \leq j \leq 5} t_{ij} \geq 0.\]
  \item[Condition 2:] Every truck has a capacity of $C$ kg.

  \[ \bigwedge_{i=1}^m  (\sum_{j=1}^5 t_{ij} \times weight(j)) \leq C .\]
  \item[Condition 3:] Every truck can carry at most eight pallets.

  \[ \bigwedge_{i=1}^m (\sum_{j=1}^5 t_{ij}) \leq M .\]
  \item[Condition 4:] The total numbers of the pallets of the other obscure building blocks should be exact the same as the given number.

  \[ \bigwedge_{1\leq j \leq 5} (\sum_{i=1}^m t_{ij}) = \sharp pallets(j) .\]

  \item[Condition 5:] Prittles and crottles are not allowed to be put in the same truck.

  \[ \bigwedge_{i=1}^m t_{i2}=0 \vee t_{i4}=0 .\]
  \item[Condition 6:] Only two of the trucks can deliver skipples.

      So let's assume the $m$-th and $m-2$-th trucks can deliver skipples.

  \[ t_{m3} + t_{(m-2)3} = M .\]
  \item[Condition 7:] No two pallets of dupples may be in the same truck.

  \[ \bigwedge_{i=1}^m t_{i5} \leq 1 .\]
\end{description}

The total formula now consists of the conjunction of all these
ingredients, that is,
\[ \bigwedge_{i,j:1 \leq i \leq m, 1 \leq j \leq 5} t_{ij} \geq 0 \;\; \wedge \]
\[ \bigwedge_{i=1}^m  (\sum_{j=1}^5 t_{ij} \times weight(j)) \leq C \;\; \wedge \]
\[ \bigwedge_{i=1}^m (\sum_{j=1}^5 t_{ij}) \leq M \;\; \wedge \]
\[ \bigwedge_{1\leq j \leq5, j\neq2} (\sum_{i=1}^m t_{ij}) = \sharp pallets(j) \;\; \wedge \]
\[ \bigwedge_{i=1}^m t_{i2}=0 \vee t_{i4}=0 \;\; \wedge \]
\[ t_{m3} + t_{(m-1)3} = M \;\; \wedge \]
\[ \bigwedge_{i=1}^m t_{i5} \leq 1 \]

This formula is easily expressed in SMT syntax. For Problem 1, we define $m = 6$, $M = 8$, $C = 7800$, and the table below.
\begin{center}
\begin{tabular}{|c|c|c|c|c|c|}
  \hline
  % after \\: \hline or \cline{col1-col2} \cline{col3-col4} ...
    & nuzzles & prittles & skipples & crottles & dupples \\
  $j$ & 1 & 2 & 3 & 4 & 5 \\
  $weight(j)$ & 700kg & 800kg & 1000kg & 1500kg & 100kg \\
  $\sharp pallets(j)$ & 4 & $n$ & 8 & 10 & 5 \\
  \hline
\end{tabular}
\end{center}

With the values above, we get the following Yices codes.

{\footnotesize

{\tt (benchmark Part1\_1.smt}

{\tt :logic $QF\_ALIA$}

{\tt :extrafuns (}

{\tt (t11 Int) (t12 Int) (t13 Int) (t14 Int) (t15 Int) }

{\tt (t21 Int) (t22 Int) (t23 Int) (t24 Int) (t25 Int) }

{\tt (t31 Int) (t32 Int) (t33 Int) (t34 Int) (t35 Int) }

{\tt (t41 Int) (t42 Int) (t43 Int) (t44 Int) (t45 Int) }

{\tt (t51 Int) (t52 Int) (t53 Int) (t54 Int) (t55 Int) }

{\tt (t61 Int) (t62 Int) (t63 Int) (t64 Int) (t65 Int) }

{\tt )}

{\tt :formula}

{\tt   (and}

{\tt (>= t11 0) (>= t12 0) (>= t13 0) (>= t14 0) (>= t15 0)}

{\tt (>= t21 0) (>= t22 0) (>= t23 0) (>= t24 0) (>= t25 0)}

$\cdots \cdots$

{\tt (>= t61 0) (>= t62 0) (>= t63 0) (>= t64 0) (>= t65 0) }

{\tt (<= (+ (* t11 700) (* t12 800) (* t13 1000) (* t14 1500) (* t15 100)) 7800)}

{\tt (<= (+ (* t21 700) (* t22 800) (* t23 1000) (* t24 1500) (* t25 100)) 7800)}

{\tt (<= (+ (* t31 700) (* t32 800) (* t33 1000) (* t34 1500) (* t35 100)) 7800)}

{\tt (<= (+ (* t41 700) (* t42 800) (* t43 1000) (* t44 1500) (* t45 100)) 7800)}

{\tt (<= (+ (* t51 700) (* t52 800) (* t53 1000) (* t54 1500) (* t55 100)) 7800)}

{\tt (<= (+ (* t61 700) (* t62 800) (* t63 1000) (* t64 1500) (* t65 100)) 7800) }

{\tt (<= (+ t11 t12 t13 t14 t15) 8)}

{\tt (<= (+ t21 t22 t23 t24 t25) 8)}

{\tt (<= (+ t31 t32 t33 t34 t35) 8)}

{\tt (<= (+ t41 t42 t43 t44 t45) 8)}

{\tt (<= (+ t51 t52 t53 t54 t55) 8)}

{\tt (<= (+ t61 t62 t63 t64 t65) 8)}

{\tt (= (+ t11 t21 t31 t41 t51 t61) 4)}

{\tt (= (+ t12 t22 t32 t42 t52 t62) 18)}

{\tt (= (+ t13 t23 t33 t43 t53 t63) 8)}

{\tt (= (+ t14 t24 t34 t44 t54 t64) 10)}

{\tt (= (+ t15 t25 t35 t45 t55 t65) 5)}

{\tt (or (= t12 0) (= t14 0)) }

{\tt (or (= t22 0) (= t24 0)) }

{\tt (or (= t32 0) (= t34 0)) }

{\tt (or (= t42 0) (= t44 0)) }

{\tt (or (= t52 0) (= t54 0)) }

{\tt (or (= t62 0) (= t64 0)) }

{\tt (or}

{\tt (= (+ t43 t63) 8) }

{\tt (<= t15 1) (<= t25 1) (<= t35 1) (<= t45 1) (<= t55 1) (<= t65 1)}

{\tt )) }
}

Applying {\tt yices-smt -m Part1\_1.smt} several times, we find when the number of pallets of prittles is 19, it is UNSAT. When the number is 18, it is SAT. Therefore, we conclude that the maximal number of pallets of prittles is 18.
The following result is yielded within a fraction of a second:

{\footnotesize

{\tt sat}

{\tt (= t52 7)}

{\tt (= t11 0)}

{\tt (= t14 5)}

{\tt (= t34 0)}

{\tt (= t21 4)}

{\tt (= t15 1)}

{\tt (= t44 2)}

{\tt (= t45 1)}

{\tt (= t54 0)}

{\tt (= t33 0)}

{\tt (= t35 0)}

{\tt (= t51 0)}

{\tt (= t23 0)}

{\tt (= t53 0)}

{\tt (= t65 1)}

{\tt (= t13 0)}

{\tt (= t62 0)}

{\tt (= t25 1)}

{\tt (= t32 8)}

{\tt (= t41 0)}

{\tt (= t63 4)}

{\tt (= t43 4)}

{\tt (= t42 0)}

{\tt (= t12 0)}

{\tt (= t64 0)}

{\tt (= t55 1)}

{\tt (= t22 0)}

{\tt (= t24 3)}

{\tt (= t61 0)}

{\tt (= t31 0)}

}

\begin{table}
  \centering
  \begin{tabular}{|l|c|c|c|c|c|}
    \hline
    % after \\: \hline or \cline{col1-col2} \cline{col3-col4} ...
     & nuzzles & prittles & skipples & crottles & dupples \\
    Truck 1 & 0 & 0 & 0 & 5 & 1 \\
    Truck 2 & 4 & 0 & 0 & 3 & 1 \\
    Truck 3 & 0 & 8 & 0 & 0 & 0 \\
    Truck 4 & 0 & 0 & 4 & 2 & 1 \\
    Truck 5 & 0 & 7 & 0 & 0 & 1 \\
    Truck 6 & 0 & 3 & 4 & 0 & 1 \\
    \hline
  \end{tabular}
\end{table}

\subsection*{Remark}
In this problem, we are required to find out the maximum number of the pallets of prittles. Such kind of maximum values searching is likely to become very time-consuming since there is no explicitly close upper bound to start the searching.

For this problem, we first find out the theoretically possible maximum number of the pallets of prittles through calculation. Because the capacity of each track is given, and it is also provided the total weights of the pallets of the other obscure building blocks, we can estimate the number.

  $\frac{7800 \times 6 - 700\times4 - 1000\times8 - 1500\times10 - 100\times5}{800} = 25.625$

Because the number of pallets is a nature number, the maximum number should not be more than 25. Therefore, we can start the debugging from $n = 25$ downwards.

\subsection*{Generalization}

We generalized all the invariants except the total number of kinds of the building blocks in this problem, because most of the requirements come from the features of different building blocks, which indicates that adding more different building blocks will generate more requirements, and as a consequent, the entire formula we summarized in the end of the solution is not applicable.

One more note is for the people who are interested in setting the number of trucks, $m$, larger than $10$. They need to care care about the notations of the variables $t_{ij}$. For instance, the number of pallets of building blocks labeled $1$ on the eleventh truck, $t_{11,1}$, is expressed as  $t111$ in Yices codes. This expression can also represent the number of pallets of building blocks labeled $11$ on the first truck, $t_{1,11}$. An extra symbol between the two numbers $i$ and $j$ is required to avoid this ambiguity.

Decreasing the number of trucks is more of interest. We did some testing to figure out how many pallets of prittles can satisfy these conditions with fewer trucks. When there are five trucks in all, ten pallets of prittles can be delivered with the satisfiability of the conditions. With fewer trucks, no satisfiability is reached. Since one truck can only deliver one pallet of drupples due to the condition, the number of trucks has to be larger than the number of pallets of drupples for delivery. It can be expressed as this
\[ \sum_{i=1}^m t_{i5}\leq m \]


\section*{Problem 3}

The goal of this problem is to exploit the power of the recommended tools rather than elaborating the questions by hand

\begin{description}
  \item[(a)] In mathematics, a \emph{group} is defined to be a set $G$ with an element $I \epsilon G$, a binary operator $\ast$ and a unary operator \emph{inv} satisfying
      \begin{center}
      $ x \ast(y \ast z)=(x\ast y) \ast z, x \ast I = x $ and $ x \ast inv(x) = I, $
      \end{center}
      for all $x, y, z \epsilon G$. Determine whether in every group each of the four properties
      \begin{center}
      $I \ast x = x, inv(inv(x)) = x, inv(x) \ast x = I $ and $ x \ast y = y \ast x $
      \end{center}
      holds for all $x, y \epsilon G$. If a property does not hold, determine the size of the smallest finite group for which it does not hold.
  \item[(b)] A term rewrite system consists of the single rule
  \[ a(x, a(y, a(z, u))) \rightarrow a(y, a(z, a(x, u)))), \]

  in which a is a binary symbol and $x, y, z, u$ are variables. Moreover, there are constants $b, c, d, e, f, g$. Determine whether $c$ and $d$ may be swapped in $a(b, a(c, a(d, a(e, a(f, a(b, g))))))$ by rewriting, that is, a(b, a(c, a(d, a(e, a(f, a(b, g)))))) rewrites in a finite number of steps to $a(b, a(d, a(c, a(e, a(f, a(b, g))))))$.

\end{description}

\vspace{4mm}

\subsection*{Solution:}

\begin{description}
  \item[(a)] In this problem, three assumptions are given. So we use $Prover9$ to prove the four properties. The codes are as follows. Here we denote \emph{inv(x)} as $x'$ for the sake of simplicity.
      
\vspace{2mm}

{\footnotesize

{\tt formulas(assumptions).}

{\tt \% Group definition}

{\tt x * I = x.}

{\tt x * x' = I.}

{\tt x * (y * z) = (x * y) * z.}

{\tt end\_of\_list.}

{\tt formulas(goals).}

{\tt I * x = x.}

{\tt x'' = x.}

{\tt x' * x = I.}

{\tt x * y = y * x.}

{\tt end\_of\_list.}

}

\vspace{2mm}

  After applying {\tt prover9 -f part2\_3a.in}, we found that the first 3 properties are proved, but the fourth one is failed. In order to determine the size of the smallest finite group for which it does not hold, we apply {\tt mace4 part$2\_3a$.in} to find the smallest noncommutative group by finding the conterexample to the statement that all groups are commutative. {\tt mace4} exits with 1 model below.
  
\vspace{2mm}
  
{\tt ============================== DOMAIN SIZE 6 =========================}

{\tt ============================== MODEL =================================}

{\tt interpretation( 6, [number=1, seconds=0], [}

{\tt \ \ \ \ \ \ \ \ function(I, [ 0 ]),}

{\tt \ \ \ \ \ \ \ \ function(c1, [ 1 ]),}

{\tt \ \ \ \ \ \ \ \ function(c2, [ 2 ]),}

{\tt \ \ \ \ \ \ \ \ function('(\_), [ 0, 1, 2, 4, 3, 5 ]),}

{\tt \ \ \ \ \ \ \ \ function(*(\_,\_), [}

{\tt \	\	\	\ \ \ \	\	\	\ \ \ \ \ \ 0, 1, 2, 3, 4, 5,}

{\tt \	\	\	\ \ \ \	\	\	\ \ \ \ \ \ 1, 0, 3, 2, 5, 4,}

{\tt \	\	\	\ \ \ \	\	\	\ \ \ \ \ \ 2, 4, 0, 5, 1, 3,}
	
{\tt \	\	\	\ \ \ \	\	\	\ \ \ \ \ \ 3, 5, 1, 4, 0, 2,}

{\tt \	\	\	\ \ \ \	\	\	\ \ \ \ \ \ 4, 2, 5, 0, 3, 1,}

{\tt \	\	\	\ \ \ \	\	\	\ \ \ \ \ \ 5, 3, 4, 1, 2, 0 ])}

{\tt ]).}

{\tt ============================== end of model ==========================}
  
\vspace{2mm}

Therefore, {\tt mace4} determines that the size of the smallest finite group, for which $ x \ast y = y \ast x $ does not hold, is 6.

  \item[(b)] To determine the possibility of the rewriting in a finite number of steps, we use $mace4$ to find a smallest number of swapping steps by finding the conterexample to the statement that $c$ and $d$ may not swap.

  This problem gives only a single rule.  Then the first step is to fulfill the assumptions with some closed terms rewriting according to the slide handout, page267. They are

      \[ R(a(x, u), a(u, x)). \]
      \[ R(x, y) \rightarrow R(a(x, z), a(y, z)). \]
      \[ R(x, y) \rightarrow R(a(z, x), a(z, y)).\]

      Then the final $mace4$ codes are

{\footnotesize

{\tt formulas(assumptions).}

{\tt R(a(x, u), a(u, x)).}

{\tt R(x, y) -> R(a(x, z), a(y, z)).}

{\tt R(x, y) -> R(a(z, x), a(z, y)).}

{\tt \% the given single rule}

{\tt RR(a(x, a(y, a(z, u))), a(y, a(z, a(x, u)))).}

{\tt end\_of\_list.}

{\tt formulas(goals).}

{\tt RR(a(b, a(c, a(d, a(e, a(f, a(b, g)))))), a(b, a(d, a(c, a(e, a(f, a(b, g))))))).}

{\tt end\_of\_list.  }

  }

  After running $mace4$, we found that the smallest number of swapping steps is 3. So $c$ and $d$ can be swapped in $a(b, a(c, a(d, a(e, a(f, a(b, g))))))$ by rewriting in 3 steps to $a(b, a(d, a(c, a(e, a(f, a(b, g))))))$.

\end{description}

\subsection*{Remark:}


\subsection*{Generalization:}






\section*{Problem 4}

Seven integer variables $a_{1}$, $a_{2}$, $a_{3}$, $a_{4}$, $a_{5}$, $a_{6}$, $a_{7}$ are given, for which the initial value of $a_{i}$ is $i$ for $i = 1, . . . , 7$. The following steps are defined: choose $i$ with $1 < i < 7$ and execute
\begin{center}
$a_{i} := a_{i - 1} + a_{i+1}$,\\
\end{center}
that is, $a_{i}$ gets the sum of the values of its neighbors and all other values remain unchanged. Show how it is possible that after a number of steps there is a number $\geq 50$ that occurs at least twice in $a_{1}$, $a_{2}$, $a_{3}$, $a_{4}$, $a_{5}$, $a_{6}$, $a_{7}$.
\vspace{4mm}

\subsection*{Solution:}

We generalize this problem by defining $n$ integer variables $a_{1}, a_{2} . . . a_{n}$, with the same initialization pattern, where $a_{i}$ is $i$ for $i = 1, . . . , n$. We also generalize the defined step, where $i$ can be chosen within the range of $1 < i < n$. The part where the selected variable gets the sum of the value of its neighbors stays the same, whereas the restriction is generalized in the form that after $m$ number of steps there is a number $\geq 50$ that occurs at least twice in $a_{1}, a_{2}, . . . , a_{n}$.

For doing so, we introduce $n\times (m+1)$ integer variables $a_{ij}$ for $i = 1, . . . , n$ and $j = 0, . . . , m$, where $a_{ij}$ represents the variable $a_{i}$ after $j$ number of steps. We also introduce $m$ integer variables $C_k$ for $k = 1, . . . , m$, where $C_k$ is the chosen index $i$ to execute the procedure for step $k$. Finally, we introduce a variable P to represent the number $\geq 50$ that we want to find after performing $m$ steps.

The problem specifies to initialize the values of $a_i$. This is expressed with the introduced variables by the formula
\[ \bigwedge_{i=1}^n (a_{i0} = i).\]

Also we have to specified the boundaries of the chosen variable, so it has to be selected within the range of $1 < i < n$ in every step. This can be expressed by the formula
\[ \bigwedge_{k=1}^m 1<C_k<n.\]

Next we express the step that after selecting a variable, it gets the sum of the values of its neighbors and all other values remain unchanged. For clarity, we split this into two conditions. One to express that the remaining variables remain with the same value, and the other to execute the sum of the neighbors. The first one can easily be expressed with the introduced variables by specifying that if $C_k$ is equal to $l$, then the values for $a_{ik}$ with $i$ different to $l$ will be the same as the values of the previous step $a_{i(k-1)}$. This is expressed with the formula
\[ \bigwedge_{k=1}^{m} \bigwedge_{l=2}^{n-1}(C_k = l) \rightarrow (\bigwedge_{i: 1 \leq i \leq n \wedge i \neq l} a_{ik} = a_{i(k-1)}).\]

Similarly, the second condition can be expressed by specifying that if $C_k$ is equal to $l$, then the value of $a_{lk}$ should be equal to the sum of its neighbors in the previous step $a_{(l-1)(k-1)}$ and $a_{(l+1)(k-1)}$. This is expressed with the formula
\[ \bigwedge_{k=1}^{m} \bigwedge_{l=2}^{n-1}(C_k = l) \rightarrow (a_{lk} = a_{(l-1)(k-1)} + a_{(l+1)(k-1)}).\]

It is worth to mention that these formulas are taking the conjunction considering $l$ running from $2$ to $n-1$. This is done this way because these are the only possible values that $C_k$ can have, as specified in formula (2).

Finally, we consider the requirement that after $m$ number of steps there is a number $P \geq 50$ that occurs at least twice. This is expressed by the formulas
\[ \bigvee_{i,i^{\prime}: 1\leq i < i^{\prime} \leq m} (a_{im} = P \wedge a_{i^{\prime} m} = P)\]
\[ P \geq 50.\]

The total formula now consists of the conjunction of all these
ingredients, that is,
\[ \bigwedge_{i=1}^n (a_{i0} = i) \;\;\wedge\]
\[ \bigwedge_{k=1}^m 1<C_k<n \;\;\wedge\]
\[ \bigwedge_{k=1}^{m} \bigwedge_{l=2}^{n-1}(C_k = l) \rightarrow (\bigwedge_{i: 1 \leq i \leq n \wedge i \neq l} a_{ik} = a_{i(k-1)}) \;\;\wedge\]
\[ \bigwedge_{k=1}^{m} \bigwedge_{l=2}^{n-1}(C_k = l) \rightarrow (a_{lk} = a_{(l-1)(k-1)} + a_{(l+1)(k-1)}) \;\;\wedge\]
\[ \bigvee_{i,i^{\prime}: 1\leq i < i^{\prime} \leq m} (a_{im} = P \wedge a_{i^{\prime} m} = P) \;\;\wedge\]
\[ P \geq 50\]

This formula can be expressed in SMT syntax, for instance, for $n=7$ and $m=10$ one can generate

{\footnotesize

{\tt (benchmark part1\_4.smt}

{\tt :logic QF\_UFLIA}

{\tt :extrafuns}

{\tt ((a1\_0 Int)  (a2\_0 Int)  (a3\_0 Int)  (a4\_0 Int)  (a5\_0 Int)  (a6\_0 Int)  (a7\_0 Int)}

{\tt (a1\_1 Int)  (a2\_1 Int)  (a3\_1 Int)  (a4\_1 Int)  (a5\_1 Int)  (a6\_1 Int)  (a7\_1 Int)}

{\tt (a1\_2 Int)  (a2\_2 Int)  (a3\_2 Int)  (a4\_2 Int)  (a5\_2 Int)  (a6\_2 Int)  (a7\_2 Int)}

{\tt (a1\_3 Int)  (a2\_3 Int)  (a3\_3 Int)  (a4\_3 Int)  (a5\_3 Int)  (a6\_3 Int)  (a7\_3 Int)}

{\tt (a1\_4 Int)  (a2\_4 Int)  (a3\_4 Int)  (a4\_4 Int)  (a5\_4 Int)  (a6\_4 Int)  (a7\_4 Int)}

{\tt (a1\_5 Int)  (a2\_5 Int)  (a3\_5 Int)  (a4\_5 Int)  (a5\_5 Int)  (a6\_5 Int)  (a7\_5 Int)}

{\tt (a1\_6 Int)  (a2\_6 Int)  (a3\_6 Int)  (a4\_6 Int)  (a5\_6 Int)  (a6\_6 Int)  (a7\_6 Int)}

{\tt (a1\_7 Int)  (a2\_7 Int)  (a3\_7 Int)  (a4\_7 Int)  (a5\_7 Int)  (a6\_7 Int)  (a7\_7 Int)}

{\tt (a1\_8 Int)  (a2\_8 Int)  (a3\_8 Int)  (a4\_8 Int)  (a5\_8 Int)  (a6\_8 Int)  (a7\_8 Int)}

{\tt (a1\_9 Int)  (a2\_9 Int)  (a3\_9 Int)  (a4\_9 Int)  (a5\_9 Int)  (a6\_9 Int)  (a7\_9 Int)}

{\tt (a1\_10 Int) (a2\_10 Int) (a3\_10 Int) (a4\_10 Int) (a5\_10 Int) (a6\_10 Int) (a7\_10 Int)}

{\tt (C1 Int)    (C2 Int)    (C3 Int)    (C4 Int)    (C5 Int)    (C6 Int)    (C7 Int)}

{\tt (C8 Int)    (C9 Int)    (C10 Int)   (P Int)}

{\tt )}

{\tt :formula}

{\tt (and }

{\tt ;The initial value of each variable is equal to its index}

{\tt (= a1\_0 1)}

{\tt (= a2\_0 2)}

{\tt (= a3\_0 3)}

{\tt (= a4\_0 4)}

{\tt (= a5\_0 5)}

{\tt (= a6\_0 6)}

{\tt (= a7\_0 7)}

{\tt ;Each choice has to be in the range of 1 to N}

{\tt (< 1 C1) (< C1 7)}

{\tt (< 1 C2) (< C2 7)}

{\tt (< 1 C3) (< C3 7)}

{\tt (< 1 C4) (< C4 7)}

{\tt (< 1 C5) (< C5 7)}

{\tt (< 1 C6) (< C6 7)}

{\tt (< 1 C7) (< C7 7)}

{\tt (< 1 C8) (< C8 7)}

{\tt (< 1 C9) (< C9 7)}

{\tt (< 1 C10) (< C10 7)}

{\tt ;If a choice is taken}

{\tt (implies (= C1 2)  (and (= a1\_1 a1\_0) (= a2\_1 (+ a1\_0 a3\_0)) (= a3\_1 a3\_0) (= a4\_1 a4\_0)}

{\tt (= a5\_1 a5\_0) (= a6\_1 a6\_0) (= a7\_1 a7\_0)))}

{\tt (implies (= C1 3)  (and (= a1\_1 a1\_0) (= a2\_1 a2\_0) (= a3\_1 (+ a2\_0 a4\_0)) (= a4\_1 a4\_0)}

{\tt (= a5\_1 a5\_0) (= a6\_1 a6\_0) (= a7\_1 a7\_0)))}

{\tt (implies (= C1 4)  (and (= a1\_1 a1\_0) (= a2\_1 a2\_0) (= a3\_1 a3\_0) (= a4\_1 (+ a3\_0 a5\_0))}

{\tt (= a5\_1 a5\_0) (= a6\_1 a6\_0) (= a7\_1 a7\_0)))}

{\tt (implies (= C1 5)  (and (= a1\_1 a1\_0) (= a2\_1 a2\_0) (= a3\_1 a3\_0) (= a4\_1 a4\_0)}

{\tt  (= a5\_1 (+ a4\_0 a6\_0)) (= a6\_1 a6\_0) (= a7\_1 a7\_0)))}

{\tt (implies (= C1 6)  (and (= a1\_1 a1\_0) (= a2\_1 a2\_0) (= a3\_1 a3\_0) (= a4\_1 a4\_0) }

{\tt (= a5\_1 a5\_0) (= a6\_1 (+ a5\_0 a7\_0)) (= a7\_1 a7\_0)))}

{\tt }

{\tt (implies (= C2 2)  (and (= a1\_2 a1\_1) (= a2\_2 (+ a1\_1 a3\_1)) (= a3\_2 a3\_1) (= a4\_2 a4\_1}

{\tt  (= a5\_2 a5\_1) (= a6\_2 a6\_1) (= a7\_2 a7\_1)))}

{\tt (implies (= C2 3)  (and (= a1\_2 a1\_1) (= a2\_2 a2\_1) (= a3\_2 (+ a2\_1 a4\_1)) (= a4\_2 a4\_1)}

{\tt  (= a5\_2 a5\_1) (= a6\_2 a6\_1) (= a7\_2 a7\_1)))}

{\tt (implies (= C2 4)  (and (= a1\_2 a1\_1) (= a2\_2 a2\_1) (= a3\_2 a3\_1) (= a4\_2 (+ a3\_1 a5\_1))}

{\tt (= a5\_2 a5\_1) (= a6\_2 a6\_1) (= a7\_2 a7\_1)))}

{\tt (implies (= C2 5)  (and (= a1\_2 a1\_1) (= a2\_2 a2\_1) (= a3\_2 a3\_1) (= a4\_2 a4\_1)}

{\tt (= a5\_2 (+ a4\_1 a6\_1)) (= a6\_2 a6\_1) (= a7\_2 a7\_1)))}

{\tt (implies (= C2 6)  (and (= a1\_2 a1\_1) (= a2\_2 a2\_1) (= a3\_2 a3\_1) (= a4\_2 a4\_1)}

{\tt (= a5\_2 a5\_1) (= a6\_2 (+ a5\_1 a7\_1)) (= a7\_2 a7\_1)))}

$\cdots \cdots$

{\tt ;After m number of steps there is a number P >= 50 that occurs at least twice }

{\tt  (or  (and (= a1\_10 P) (= a2\_10 P))}

{\tt (and (= a1\_10 P) (= a3\_10 P))}

 {\tt (and (= a1\_10 P) (= a4\_10 P))}

 {\tt (and (= a1\_10 P) (= a5\_10 P))}

 {\tt (and (= a1\_10 P) (= a6\_10 P))}

 {\tt (and (= a1\_10 P) (= a7\_10 P))}

 {\tt (and (= a2\_10 P) (= a3\_10 P))}

 {\tt (and (= a2\_10 P) (= a4\_10 P))}

 {\tt (and (= a2\_10 P) (= a5\_10 P))}

 {\tt (and (= a2\_10 P) (= a6\_10 P))}

 {\tt (and (= a2\_10 P) (= a7\_10 P))}

$\cdots \cdots$

{\tt (>= P 50)}

{\tt )}
}

\vspace{3mm}

Applying {\tt yices-smt -m part$1\_4$.smt}, it yields the following results:

{\footnotesize

{\tt (= a1\_0 1)}

{\tt (= a2\_0 2)}

{\tt (= a3\_0 3)}

{\tt (= a4\_0 4)}

{\tt (= a5\_0 5)}

{\tt (= a6\_0 6)}

{\tt (= a7\_0 7)}

$\cdots \cdots$

{\tt (= a1\_10 1)}

{\tt (= a2\_10 4)}

{\tt (= a3\_10 57)}

{\tt (= a4\_10 53)}

{\tt (= a5\_10 50)}

{\tt (= a6\_10 57)}

{\tt (= a7\_10 7)}

{\tt (= C1 4)}

{\tt (= C2 2)}

{\tt (= C3 6)}

{\tt (= C4 5)}

{\tt (= C5 6)}

{\tt (= C6 4)}

{\tt (= C7 5)}

{\tt (= C8 4)}

{\tt (= C9 3)}

{\tt (= C10 6)}

{\tt (= P 57)}

}

\vspace{3mm}

The final result is shown in next table.

\begin{center}
\begin{tabular}{|c|c|c|c|c|c|c|c|c|c|c|c|}
  \hline
  $variables/step$ & 0 & 1 & 2 & 3 & 4 & 5 & 6 & 7 & 8 & 9 & 10 \\
  \hline
  $C_{k}$ &   & 4 & 2 & 6 & 5 & 6   & 4  & 5  & 4 & 3 & 6 \\
  $a_{1}$ & 1 & 1 & 1 & 1 & 1 & 1   & 1  & 1  & 1 & 1 & 1 \\
  $a_{2}$ & 2 & 2 & 4 & 4 & 4 & 4   & 4  & 4  & 4 & 4 & 4 \\
  $a_{3}$ & 3 & 3 & 3 & 3 & 3 & 3   & 3  & 3  & 3 & 57 & \textbf{57} \\
  $a_{4}$ & 4 & 8 & 8 & 8 & 8 & 8   & 23 & 23 & 53 & 53 & 53 \\
  $a_{5}$ & 5 & 5 & 5 & 5 & 20 & 20 & 20 & 50 & 50 & 50 & 50 \\
  $a_{6}$ & 6 & 6 & 6 & 12 & 12 & 27 & 27 & 27 & 27 & 27 & \textbf{57} \\
  $a_{7}$ & 7 & 7 & 7 & 7 & 7 & 7 & 7 & 7 & 7 & 7 & 7 \\
  \hline
\end{tabular}
\end{center}

As can be seen, after 10 steps, it is possible to find a number $\geq 50$ that occurs at least twice. For this case this number is 57.

\subsection*{Remark:}

Although the method presented here successfully solves the problem, it is not very practical for large number of steps. The reason is that for every extra step that we want to analyze, it is necessary to add $n+1$ extra variables to the yices code, with its respective conditions, resulting in a significant increment of code. Additionally, since the steps have to be explicitly added, if for some reason such number of steps $m$ does not exist (e.g. choosing n = 3), then no matter how many steps we added to the code, we will never get a satisfied condition.

\subsection*{Generalization:}

We solved this problem choosing the number of variables $n = 7$, but it would be interesting to know the results for other values of $n$.  For $n>7$ it is easy to see that the formula is still satisfiable after 10 steps, since we can choose the same $C_{k}$ variables as for $n=7$ and the result will be the same.
For $n = 3$ the resulting formula is unsatisfiable no matter how many number of steps we choose, this can be easily seen because we only can choose the variable $a_{2}$ to execute the steps for this specific case.

For $n=4$ the formula again is always unsatisfiable. We can see this by observing the facts that we only can choose $a_{2}$ and $a_{3}$ to execute the steps, and that it will always be the case that $a_{2} = 1 + a_{3}$ or $a_{3} = 4 + a{2}$, so they will never be equal. For $n=5$ and $n=6$, and considering the number of steps $m=10$, the formula is unsatisfiable. 

\end{document}
